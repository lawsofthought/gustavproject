\section{The probabilistic model}

One of the most straightforward applications of the multinomial data \hdpmm is
as a \emph{bag-of-words} probabilistic language model, and in what follows
we'll describe it with this application in mind. However, modulo some possible
changes in notation, this will in fact constitute a general description of the
\hdpmm for multinomial data.

In general, according to a bag-of-words language model, a corpus of natural
language is a set of $J$ documents or texts \[w_1, w_2 \ldots w_j \ldots w_J,\]
where text $j$, i.e., $w_j$, is a set of $n_j$ words from a finite vocabulary
of $V$ word types. For simplicity, this vocabularly can be represented as the
$V$ integers $\{1, 2 \ldots V\}$.  From this, we have each $w_j$ defined as
\[w_j = w_{j1}, w_{j2} \ldots w_{ji} \ldots w_{jn_j},\] with each $w_{ji} \in
\{1 \ldots V\}$.  The \emph{bag-of-words} assumption is that, for each text,
$w_{j1}, w_{j2} \ldots w_{ji} \ldots w_{jn_j}$ are exchangeable random
variables, i.e.  their joint probability distribution is invariant to any
permutation of the indices.  By this assumption therefore, as the name implies,
each text is modelled as an unordered set, or \emph{bag}, of words.

As a generative model of this language corpus, the \textsc{hdpmm} treats each
observed word $w_{ji}$ as a sample from one of an underlying set of text or
discourse topics: \[ \phi = \phi_1, \phi_2 \ldots \phi_k \ldots, \] where each
$\phi_k$ is a probability distribution over $\{1 \ldots V\}$.  The identity of
the particular topic distribution from which $w_{ji}$ is drawn is determined by
the value of a discrete latent variable $x_{ji} \in \{1, 2 \ldots k \ldots\}$
that corresponds to $w_{ji}$. As such, for each $w_{ji}$, we model it as
\[w_{ji}\vert x_{ji}, \phi \sim \dcat{\phi_{x_{ji}}}.\] 

To be clear, the \hdpmm assumes that there are an unlimited number of topic
distributions from which the observed data are drawn, and so each $x_{ji}$ can
take on infinitely many discrete values.  The probability distribution over the
infinitely possible values of each $x_{ji}$ is given by an infinite length
categorical distribution $\pi_j$, i.e., $\pi_j = \pi_{j1}, \pi_{j2} \ldots
\pi_{jk} \ldots$, where $0 \leq \pi_{jk} \leq 1$ and $\sum_{k=1}^\infty
\pi_{jk} = 1$, that is specific to text $j$.  In other words, \[x_{ji} \given
\pi_j \sim \dcat{\pi_j}.\] Each $\pi_j$ is assumed to be drawn from a Dirichlet
process prior whose base distribution, $m$, is a categorical distribution over
the positive integers and whose scalar concentration parameter is $a$: 
\[\pi_j\vert a,m \sim \ddp{a,m}.\]
The $m$ base distribution is assumed to be drawn from a stick breaking
distribution with a parameter $\gamma$: \[ m\vert\gamma \sim \dstick{\gamma}.
\]

The prior distributions of the Dirichlet process concentration parameter $a$
and the stick-breaking parameter $\gamma$ are Gamma distributions, both with
shape and scale parameters equal to 1. For the topic distributions, $\phi_1,
\phi_2 \ldots \phi_k \ldots$, we can assume they are independently and
identically drawn from a Dirichlet distribution with a length $V$ location
parameter $\psi$ and concentration parameter $b$. In turn, $\psi$ is drawn from
a symmetric Dirichlet distribution with concentration parameter $c$. Finally,
both $b$ and $c$, like $a$ and $\gamma$, can be given Gamma priors, again with
shape and scale distributions equal to 1.
